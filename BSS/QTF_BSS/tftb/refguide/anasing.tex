% This is part of the TFTB Reference Manual.
% Copyright (C) 1996 CNRS (France) and Rice University (US).
% See the file refguide.tex for copying conditions.



\markright{anasing}
\section*{\hspace*{-1.6cm} anasing}

\vspace*{-.4cm}
\hspace*{-1.6cm}\rule[0in]{16.5cm}{.02cm}
\vspace*{.2cm}



{\bf \large \sf Purpose}\\
\hspace*{1.5cm}
\begin{minipage}[t]{13.5cm}
Lipschitz singularity.
\end{minipage}
\vspace*{.5cm}


{\bf \large \sf Synopsis}\\
\hspace*{1.5cm}
\begin{minipage}[t]{13.5cm}
\begin{verbatim}
x = anasing(N)
x = anasing(N,t0)
x = anasing(N,t0,H)
\end{verbatim}
\end{minipage}
\vspace*{.5cm}


{\bf \large \sf Description}\\
\hspace*{1.5cm}
\begin{minipage}[t]{13.5cm}
        {\ty anasing} generates the N-points Lipschitz singularity centered
        around {\ty t0} : $x(t) = |t-t0|^H$.\\

\hspace*{-.5cm}\begin{tabular*}{14cm}{p{1.5cm} p{8.5cm} c}
Name & Description & Default value\\
\hline
        {\ty N}  & number of points in time\\
        {\ty t0} & time localization of the singularity  & {\ty N/2}\\
        {\ty H}  & strength of the Lipschitz singularity (positive or
             negative)                             & {\ty 0}\\
  \hline {\ty x}  & the time row vector containing the signal samples\\
\hline
\end{tabular*}

\end{minipage}
\vspace*{1cm}


{\bf \large \sf Example}
\begin{verbatim}
         x=anasing(128); plot(real(x));
\end{verbatim}
\vspace*{.5cm}


{\bf \large \sf See Also}\\
\hspace*{1.5cm}
\begin{minipage}[t]{13.5cm}
\begin{verbatim}
anastep, anapulse, anabpsk, doppler, holder.
\end{verbatim}
\end{minipage}
\vspace*{.5cm}


{\bf \large \sf Reference}\\
\hspace*{1.5cm}
\begin{minipage}[t]{13.5cm}
[1] S. Mallat and W.L. Hwang ``Singularity Detection and Processing with
Wavelets'' IEEE Trans. on Information Theory, Vol 38, No 2, March 1992,
pp. 617-643.
\end{minipage}


% This is part of the TFTB Reference Manual.
% Copyright (C) 1996 CNRS (France) and Rice University (US).
% See the file refguide.tex for copying conditions.


\markright{tfrview}
\section*{\hspace*{-1.6cm} tfrview}

\vspace*{-.4cm}
\hspace*{-1.6cm}\rule[0in]{16.5cm}{.02cm}
\vspace*{.2cm}

{\bf \large \sf Purpose}\\
\hspace*{1.5cm}
\begin{minipage}[t]{13.5cm}
Visualization of time-frequency representations.
\end{minipage}
\vspace*{.5cm}

{\bf \large \sf Synopsis}\\
\hspace*{1.5cm}
\begin{minipage}[t]{13.5cm}
\begin{verbatim}
tfrview(tfr,sig,t,method,param,map) 
tfrview(tfr,sig,t,method,param,map,p1) 
tfrview(tfr,sig,t,method,param,map,p1,p2) 
tfrview(tfr,sig,t,method,param,map,p1,p2,p3) 
tfrview(tfr,sig,t,method,param,map,p1,p2,p3,p4) 
tfrview(tfr,sig,t,method,param,map,p1,p2,p3,p4,p5) 
\end{verbatim}
\end{minipage}
\vspace*{.5cm}

{\bf \large \sf Description}\\
\hspace*{1.5cm}
\begin{minipage}[t]{13.5cm}
        {\ty tfrview} visualizes a time-frequency representation. It is
        called through {\ty tfrqview} from any {\ty tfr*} function when
        this function is called without output argument. {\bf Use {\ty
        tfrqview} preferably}.\\

\hspace*{-.5cm}\begin{tabular*}{14cm}{p{1.5cm} p{11.5cm}}
Name & Description\\
\hline
        {\ty tfr}    & time-frequency representation\\
        {\ty sig}    & signal in the time-domain\\
        {\ty t}      & time instants\\
        {\ty method} & chosen representation (name of the corresponding M-file)\\
        {\ty param}  & visualization parameter vector :\\
         &  {\ty param = [display linlog threshold levnumb nf2} ... \\
			& \hspace*{2cm} {\ty layout access state fs isgrid]} where\\ 
          & - {\ty display=1..5} for {\ty contour, imagesc, pcolor, surf} or {\ty mesh}\\ 
          & - {\ty linlog=0/1} for linearly/logarithmically
			spaced levels for the amplitude of {\ty tfr}\\ 
          & - {\ty threshold}  is the visualization threshold, in \% \\
          & - {\ty levelnumb}  is the number of levels used with {\ty contour}\\
          & - {\ty nf2}        is the number of frequency bins displayed\\ 
          & - {\ty layout} determines the layout of the figure\,: {\ty
		tfr} alone (1), {\ty tfr} and {\ty sig} (2), {\ty tfr} and
		spectrum (3), {\ty tfr} and {\ty sig} and spectrum (4), add/remove
		the colorbar (5)\\ 
          & - {\ty access} depends on the way you access to {\ty
	tfrview}\,: from the command line (0) ; from {\ty tfrqview}, except
	after a change in the sampling frequency or in the layout (1) ;
	from {\ty tfrqview}, after a change in the layout (2) ; 
	from {\ty tfrqview}, after a change in the sampling frequency (3)\\  

\hline\end{tabular*}\end{minipage} 
%\newpage
\hspace*{1.5cm}\begin{minipage}[t]{13.5cm}
\hspace*{-.5cm}\begin{tabular*}{14cm}{p{1.5cm} p{8.5cm} c}
Name & Description & Default value\\\hline

          & - {\ty state} depends on the signal/colorbar presence\,:
		no signal, no colorbar (0) ; signal, no colorbar (1) ; no
		signal, colorbar (2) ; signal and colorbar (3)\\ 
          & - {\ty fs} is the sampling frequency\\ 
          & - {\ty isgrid} depends on the grids' presence\,:\\ & {\ty
		isgrid=isgridsig+2*isgridspe+4*isgridtfr}\\ & where {\ty isgridsig=1} if
		a grid is present on the signal and {\ty =0} if not, and so on\\ 
        {\ty map}    & selected colormap\\ 
        {\ty p1..p5} & parameters of the representation. Run 
                 {\ty tfrparam(method)} to know the meaning of {\ty p1..p5}\\

\hline
\end{tabular*}

\end{minipage}
\vspace*{1cm}

{\bf \large \sf See Also}\\
\hspace*{1.5cm}
\begin{minipage}[t]{13.5cm}
\begin{verbatim}
tfrqview, tfrparam, tfrsave.
\end{verbatim}
\end{minipage}


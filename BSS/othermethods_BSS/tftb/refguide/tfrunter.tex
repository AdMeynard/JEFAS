% This is part of the TFTB Reference Manual.
% Copyright (C) 1996 CNRS (France) and Rice University (US).
% See the file refguide.tex for copying conditions.


\markright{tfrunter}
\section*{\hspace*{-1.6cm} tfrunter}

\vspace*{-.4cm}
\hspace*{-1.6cm}\rule[0in]{16.5cm}{.02cm}
\vspace*{.2cm}

{\bf \large \sf Purpose}\\
\hspace*{1.5cm}
\begin{minipage}[t]{13.5cm}
Unterberger time-frequency distribution, active or passive form.
\end{minipage}
\vspace*{.5cm}

{\bf \large \sf Synopsis}\\
\hspace*{1.5cm}
\begin{minipage}[t]{13.5cm}
\begin{verbatim}
[tfr,t,f] = tfrunter(x)
[tfr,t,f] = tfrunter(x,t)
[tfr,t,f] = tfrunter(x,t,form)
[tfr,t,f] = tfrunter(x,t,form,fmin,fmax)
[tfr,t,f] = tfrunter(x,t,form,fmin,fmax,N)
[tfr,t,f] = tfrunter(x,t,form,fmin,fmax,N,trace)
\end{verbatim}
\end{minipage}
\vspace*{.5cm}

{\bf \large \sf Description}\\
\hspace*{1.5cm}
\begin{minipage}[t]{13.5cm}
        {\ty tfrunter} generates the auto- or cross-Unterberger
        distribution (active or passive form). The expression of the active
Unterberger distribution writes
\begin{eqnarray*}
U^{(a)}_x(t,a)=\frac{1}{|a|}\ \int_0^{+\infty} (1+\frac{1}{\alpha^2})\
X\left(\frac{\alpha}{a}\right)\  X^*\left(\frac{1}{\alpha a}\right)\
e^{j2\pi (\alpha-1/\alpha)\frac{t}{a}}\ d\alpha,       
\end{eqnarray*}
whereas the passive Unterberger distribution writes
\begin{eqnarray*}
U^{(p)}_x(t,a)=\frac{1}{|a|} \int_0^{+\infty} \frac{2}{\alpha}\
X\left(\frac{\alpha}{a}\right)\ X^*\left(\frac{1}{\alpha a}\right)\
e^{j2\pi (\alpha-\frac{1}{\alpha})\frac{t}{a}}\ d\alpha.       
\end{eqnarray*}

\hspace*{-.5cm}\begin{tabular*}{14cm}{p{1.5cm} p{8.5cm} c}
Name & Description & Default value\\
\hline
        {\ty x} & signal (in time) to be analyzed. If {\ty x=[x1 x2]}, {\ty tfrunter} 
           computes the cross-Unterberger distribution {\ty (Nx=length(x))}\\
        {\ty t} & time instant(s) on which the {\ty tfr} is evaluated & {\ty (1:Nx)}\\
        {\ty form} & {\ty 'A'} for active, {\ty 'P'} for passive
			Unterberger distribution     & {\ty 'A'}\\
        {\ty fmin, fmax} & respectively lower and upper frequency bounds of 
           the analyzed signal. These parameters fix the equivalent 
           frequency bandwidth (expressed in Hz). When unspecified, you
           have to enter them at the command line from the plot of the
           spectrum. {\ty fmin} and {\ty fmax} must be $>${\ty 0} and $\leq${\ty 0.5}\\
        {\ty N} & number of analyzed voices & auto\footnote{This value,
	determined from {\ty fmin} and {\ty fmax}, is the 
	next-power-of-two of the minimum value checking the non-overlapping
	condition in the fast Mellin transform.}\\

\hline\end{tabular*}\end{minipage} 
%\newpage
\hspace*{1.5cm}\begin{minipage}[t]{13.5cm}
\hspace*{-.5cm}\begin{tabular*}{14cm}{p{1.5cm} p{8.5cm} c}
Name & Description & Default value\\\hline

        {\ty trace} & if nonzero, the progression of the algorithm is shown
                                                & {\ty 0}\\
        \hline {\ty tfr} & time-frequency matrix containing the coefficients of the 
           decomposition (abscissa correspond to uniformly sampled
           time, and ordinates correspond to a geometrically sampled 
           frequency). First row of {\ty tfr} corresponds to the lowest 
           frequency. \\
        {\ty f} & vector of normalized frequencies (geometrically sampled 
           from {\ty fmin} to {\ty fmax})\\

\hline
\end{tabular*}
\vspace*{.2cm}

When called without output arguments, {\ty tfrunter} runs {\ty tfrqview}.
\end{minipage}
\vspace*{1cm}


{\bf \large \sf Example}
\begin{verbatim}
         sig=altes(64,0.1,0.45); 
         tfrunter(sig);
\end{verbatim}
\vspace*{.5cm}


{\bf \large \sf See Also}\\
\hspace*{1.5cm}
\begin{minipage}[t]{13.5cm}
all the {\ty tfr*} functions.
\end{minipage}
\vspace*{.5cm}

{\bf \large \sf References}\\
\hspace*{1.5cm}
\begin{minipage}[t]{13.5cm}
[1] A. Unterberger ``The Calculus of Pseudo-Differential Operators of Fuchs
Type'' Comm. in Part. Diff. Eq., Vol. 9, pp. 1179-1236, 1984.\\

[2] P. Flandrin, P. Gon�alv�s ``Geometry of Affine Time-Frequency
Distributions'' Applied and Computational Harmonic Analysis, Vol. 3,
pp. 10-39, January 1996.
\end{minipage}

